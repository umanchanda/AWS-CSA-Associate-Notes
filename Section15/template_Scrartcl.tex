\documentclass[]{scrartcl}

%opening
\title{CloudFront and AWS Global Accelerator}
\author{Uday Manchanda}

\begin{document}

\maketitle

\section{CloudFront}

\begin{itemize}
	\item CDN (Content Delivery Network)
	\item Improves read performance, content is cached at the edge
	\item 216 edge locations
	\item DDos Protection, integration with AWS Shield, AWS WAF
	\item Expose external HTTPS and can talk to internal HTTPS backends
	\item Origins
	\begin{itemize}
		\item S3 bucket - for distributing files and caching them at the edge
		\item enhanced security with CF Origin Access Identity (OAI) - allows communication from CF and nowhere else
		\item Custom origin (HTTP) - ALB, EC2, S3 website, any HTTP backend
	\end{itemize}
	\item Edge location are connected to the origin we defined
	\item Client access CF distribution, client sends HTTP request to CF, edge location forwards request to your origin
	\item Edge location caches the request
\end{itemize}

\subsection{CF - S3 as an origin}
\begin{itemize}
	\item edge location will fetch data from s3 location
	\item edge location will use an OAI to access s3 bucket, which is an IAM role for your CF origin
\end{itemize}

\subsection{ALB or EC2 as an origin}
\begin{itemize}
	\item Ec2 instances must be public
	\item ege location will access ec2 instance, must allow public IP of edge location
	\item for alb it also must be public
	\item alb must allow public ip of edge location, alb then allows SG of of LB
\end{itemize}

\subsection{Geo Restriction}
\begin{itemize}
	\item you can restrict who can access your distribution
	\item via: whitelist, blacklist
	\item Country is determined via third party geo ip database
	\item CF vs s3 CRR
	\begin{itemize}
		\item Cf - great for static content that must be available everywhere
		\item s3 crr - great for dynamic content that needs to be available at low-latency in a few regions
	\end{itemize}
\end{itemize}

\subsection{CloudFront Signed URL/Cookies}
\begin{itemize}
	\item lets say you want to distribute paid shared content to premium users all over the world
	\item attach a policy with: url expiration, IP ranges, trusted signers
	\item Signed URL = access to individual files (one signed URL per file)
	\item Signed cookies = access to multiple files
	\item CF signed url vs S3 signed URL
	\begin{itemize}
		\item CF - allows access to a path, no matter the origin, account wide, leverge all caching features
		\item s3 - issue a request as the person who pre signed the URL, uses IAM key of the signing IAM principal, limited lifetime
	\end{itemize}
\end{itemize}

\subsection{Advanced Concepts}
\begin{itemize}
	\item the more data transferred out of CF the lower the cost
	\item ccan reduce the number of edge locations for cost reduction
	\item three price classes
	\begin{itemize}
		\item Price class all
		\item Price class 200
		\item Price class 100
	\end{itemize}
	\item Multiple Origin - route to different kinds of origins based on the content type based on path pattern
	\item Origin Groups - increase HA and do failover. Have one primary and one secondary
	\item can set up replication if primary and secondary buckets are in different regions
	\item Field level encryption - protect sensitive information thru application stack
	\item sensitive info encrypted at the edge location close to the user
	\item Uses assyemtric encryption
\end{itemize}

\section{Global Accelerator}
\begin{itemize}
	\item The problem
	\begin{itemize}
		\item Deployed a global application and have global users who want to access it
		\item they go over the public internet which can add a lot of latency due to hops
		\item we wish to go as fast as possible thru aws newtork to minimize latency
	\end{itemize}
	\item uses anycast IP to send traffic directly to edge locations, edge locations send traffic to your application
	\item leverages the aws internal network to route your application
	\item works with: elastic IP, ec2, ALb, NLB, public or private
	\item Consistent Performance - intelligent routing, no issues with caching, internal network
	\item Health checks
	\item only 2 ips need to be whitelisted and DDoS protection
\end{itemize}

\subsection{Global Accelerator vs CloudFront}

\begin{itemize}
	\item both use the aws global network and its edge locations around the world
	\item both integrate with aws shield for DDoS protection
	\item CloudFront
	\begin{itemize}
		\item improves performance for both cacheable content and dynamic content
		\item content is served at the edge
	\end{itemize}
	\item Global Accelerator
	\begin{itemize}
		\item improves performance for a wide range of applications over TCP or UDP
		\item Proxying packets at the edge to applications running in one or more AWS regions
		\item Good fit for non HTTP use cases
		\item or for HTTP use cases that require static IP addresses or fast failover
	\end{itemize}
\end{itemize}

\end{document}
