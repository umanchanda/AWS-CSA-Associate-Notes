\documentclass[]{scrartcl}

%opening
\title{AWS Fundamentals: RDS + Aurora + ElastiCache}
\author{Uday Manchanda}

\begin{document}

\maketitle

\section{RDS}
\begin{itemize}
	\item Relational Database Service
	\item Managed DB service for DBs that use SQL
	\item Postgres, MySQL, MariaDB, Oracle, Microsoft SQL
	\item Using RDS vs deploying DB on EC2
	\begin{itemize}
		\item Managed service: automated provisioning, OS patching
		\item Continous Backups
		\item Monitoring dashboards
		\item Read replicas
		\item Multi AZ setup for DR
		\item Scaling capability
		\item storage backed by EBS
		\item Cannot SSH into your instances
	\end{itemize}
	\item RDS Backups
	\begin{itemize}
		\item Backups are automatically enabled in RDS
		\item Automated Backups: daily full backup, transaction logs, 7 days retention
	\end{itemize}
\end{itemize}

\subsection{Storage Auto Scaling}
\begin{itemize}
	\item Helps increase storage on RDS DB instance dynamically
	\item If RDS detects you are running out of free DB storage, it scales automatically
	\item Useful for apps with unpredictable workloads
	\item Supports all RDS DB engines
\end{itemize}

\subsection{Read replicas vs Multi AZ}
\begin{itemize}
	\item Read Replicas for read stability
	\begin{itemize}
		\item Up to 5 read replicas
		\item Within AZ, Cross AZ, or Cross Region
		\item is Async, so reads are eventually consistent
		\item Replicas can be promoted to their own DB
		\item Applications must update the connection string to leverage read replicas
	\end{itemize}
	\item RR Use case
	\begin{itemize}
		\item Prod DB taking on normal load
		\item You want to run a reporting application to run some analytics
		\item Create RR to run the new workload there
		\item Prod DB is unaffected
		\item RR are used for SELECT type kind of statements (no INSERT/UPDATE etc)
	\end{itemize}
	\item RR Network Cost
	\begin{itemize}
		\item There's a network cost when data goes from one AZ to another
		\item For RDS read replicas within the same region, you don't pay that fee
		\item ASYNC replication: For example, if your application reads from the RR before they had the chance to to replicate the data then you may get all the data. Called $\textit{eventually consistent asynchronous replication}$
	\end{itemize}
	\item RDS Multi AZ (Disaster Recovery)
	\begin{itemize}
		\item SYNC replication
		\item EX: RDS DB in AZ A and RDS DB standby in AZ B
		\item When your application writes to the master, that change needs to be replicated to the standby to be accepted
		\item One DNS name
		\item If there is a problem with master, automatic failover to standby
		\item that standby will get promoted to master
		\item Not used for scaling
		\item Useful for disaster recovery
	\end{itemize}
	\item RDS from Single AZ to Multi AZ
	\begin{itemize}
		\item Zero downtime operation
		\item click on modify for DB, enable multi AZ
		\item What happens internally
		\begin{itemize}
			\item A snapshot is taken
			\item New DB is restored from the snapshot in a new AZ
			\item Synchronization is established between the two databases
		\end{itemize}
	\end{itemize}
\end{itemize}

\subsection{Encryption and Security}
\begin{itemize}
	\item At rest encryption: data not in movement
	\begin{itemize}
		\item Encrypt master and RR with AWS KMS (key management services)
		\item Must be defined at launch time
		\item If the master is not encrypted, neither can the RRs
	\end{itemize}
	\item In flight encryption
	\begin{itemize}
		\item SSL certificates to encrypt data to RDS in flight
		\item Provide SSL options with trust certificate when connecting to DB
	\end{itemize}
	\item Encryption Operations
	\begin{itemize}
		\item Snapshots of encrypted RDS DBs are encrypted and vice-versa
		\item Can copy a snapshot into an encrypted one
		\item To encrypt an un-encrypted DB
		\begin{itemize}
			\item Create snapshot of the un-encrypted DB
			\item Copy the snapshot and enable encryption for the snapshot
			\item Restore the DB from the encrypted snapshot
			\item Migrate applications to the new DB and delete the old one
		\end{itemize}
		\item RDS DBs are deployed within private subnet
		\item Security works by leveraging security groups
		\item Access mgt handled with IAM policies, username/password
		\item RDS IAM authentication
		\begin{itemize}
			\item works with mysql and postgres
			\item Don't need a password, just an authentication token obtained through IAM and RDS API calls
			\item Token has lifetime of 15 mins
			\item Easier to centrally manage users/roles with IAM
		\end{itemize}
	\end{itemize}
\end{itemize}

\section{Aurora}
\begin{itemize}
	\item Proprietary technology from AWS
	\item Postgres and MySQL are supported
	\item "Cloud optimized" - better performance than RDS
	\item Storage grows in 10 gb increments
	\item Failover is instantaneous. High Availability
\end{itemize}

\subsection{High Availability and Read Scaling}
\begin{itemize}
	\item 6 copies of data across 3 AZ (4 needed for writes, 3 for reads)
	\item Self healing, striped storage
	\item One Aurora instance takes writes
	\item Automated failover very quick
	\item One master and up to 15 RRs
	\item Support for Cross Region Replication
\end{itemize}

\subsection{DB Cluster}
\begin{itemize}
	\item Master writes to shared storage volume (which expands)
	\item Writer endpoint always points to master, useful if autofailing. It's a DNS name
	\item Even if master fails, the clients can talk to the writer endpoint and is automatically redirected to the right instance
	\item Can set up autoscaling on your RRs
	\item Reader endpoint - helps with connection load balancing, connects to all your RRs
	\item Backtrack: restore data at any point without using backups
	\item Usually pick either: one writer with multiple readers or serverless (expand/contract based on demand)
	\item Writers and readers are separate
	\item You should either select the writer endpoint to write or a reader endpoint to read
\end{itemize}

\subsection{Security}
\begin{itemize}
	\item Encryption at rest with KMS
	\item In-flight with SSL
	\item Pretty much the same as RDS
\end{itemize}

\subsection{Advanced Concepts}
\begin{itemize}
	\item Replica autoscaling
	\begin{itemize}
		\item If there is increased load on the reader endpoints
		\item Replica autoscaling will allow more reader endpoints to be created
	\end{itemize}
	\item Custom Endpoints
	\begin{itemize}
		\item Define a subset of aurora instances as a custom endpoint
		\item EX: have two larger RRs to run analytic queries on
		\item Point those RRs to the custom endpoint, not the reader endpoint
		\item Set up several custom endpoints for different types of workloads
	\end{itemize}
	\item Serverless
	\begin{itemize}
		\item Automated DB instantiation and auto scaling based on actual usage
		\item Useful for unpredictable workloads
		\item No capacity planning required
	\end{itemize}
	\item Multi-Master
	\begin{itemize}
		\item If you want immediate failover for write node (HA)
		\item Every node does R/W vs promoting a RR as the new master
	\end{itemize}
	\item Global Aurora
	\begin{itemize}
		\item Aurora Global DB: 1 primary region, up to 5 secondary regions, up to 16 RR per secondary region
	\end{itemize}
	\item Can add ML based predictions to your applications via SQL (SageMaker/Comprehend)
\end{itemize}

\section{ElastiCache}
\begin{itemize}
	\item The way to get managed Redis or memcached
	\item Caches are in-memory DBs with really high performance, low latency
	\item Reduces load off DBs
	\item Makes application stateless
	\item Involves heavy application code changes
\end{itemize}

\subsection{Solution Architecture}
\begin{itemize}
	\item DB Cache
	\begin{itemize}
		\item Application queries ElastiCache, which sees if the query has already been made
		\item if its already been made - cache hit, gets the answer straight from elasticache
		\item if not - cache miss. We need to fetch the data from the DB and then write to elasticache
		\item relieves load in RDS
	\end{itemize}
	\item user session store
	\begin{itemize}
		\item makes app stateless
		\item app writes session data to cache
		\item If the user is re-directed, retrieve session from elasticache
	\end{itemize}
\end{itemize}

\subsection{Redis vs Memcached}
\begin{itemize}
	\item Redis
	\begin{itemize}
		\item Multi AZ with autofailover
		\item Read replicas to scale reads and have high availability
		\item Data durability
		\item Sorta like RDS
	\end{itemize}
	\item Memcached
	\begin{itemize}
		\item Multi node for partitioning of data (sharding)
		\item no HA, non persistent, no backup
		\item Multi-threaded architecture
	\end{itemize}
\end{itemize}

\subsection{Cache Security}
\begin{itemize}
	\item Does not support IAM authentication
	\item Redis Auth (SSL in flight)
\end{itemize}

\subsection{Patterns}
\begin{itemize}
	\item Lazy loading: all the read data is cached, data can become stale in cache
	\item Write through: adds or update data in the cache when written to a DB (no stale data)
	\item Session store: store temp session data in a cache
\end{itemize}

\subsection{Use Cases}
\begin{itemize}
	\item Gaming leaderboard
	\begin{itemize}
		\item Redis sorted sets guarantee uniqueness and element ordering
		\item When a new element is added, its ranked in real time, then added in correct order
	\end{itemize}
\end{itemize}


\end{document}
