\documentclass[]{scrartcl}

%opening
\title{Containers on AWS: ECS, Fargate, ECR, and EKS}
\author{Uday Manchanda}

\begin{document}

\maketitle

\section{ECS}
\begin{itemize}
	\item elastic container service, allows you to launch docker containers on AWS
	\item must provision and maintain the infrastructure (EC2 instances)
	\item AWS takes care of starting and stopping containers
	\item integrates with ALB to expose application to the web
	\item Launch Types:
	\begin{itemize}
		\item EC2 launch type for ECS: 
		\begin{itemize}
			\item ECS cluster in a single VPC
			\item Create ASG to create EC2 instances aka container instances. 
			\item Run ECS agent on all the instances, register those instances to the ECS cluster, and start launching some tasks. 
			\item Each EC2 instance can run one or more tasks
		\end{itemize}
		\item Fargate. 
		\begin{itemize}
			\item No EC2 instances to provision. Serverless offering
			\item AWS just runs the containers for you based on CPU/RAM needs. 
			\item Fargate will launch a task
			\item To access the task, we will have an ENI (elastic network interface) launched within the VPC to bind this task to a network IP. 
			\item More tasks = more ENIs. Each ENI is a distinct IP address
		\end{itemize}
	\end{itemize}
\end{itemize}

\subsection{IAM Roles for ECS Tasks}
\begin{itemize}
	\item ECS tasks might need to interact with other AWS services - will need IAM roles for them
	\item EC2 instance profile - used by ECS agent, makes API calls to ECS service, send container logs
	\item ECS task role:
	\begin{itemize}
		\item allows each task to have a specific role
		\item use different roles for the different ECS services you run
		\item task role is defined in the task definition
		\item task roles allow, for example, access to an s3 bucket
	\end{itemize}
\end{itemize}

\subsection{Data Volumes - EFS File Systems}
\begin{itemize}
	\item if you want to share data: create EFS file system and mount it directly on the ECS tasks.
	\item works for both EC2 and fargate tasks. 
\end{itemize}

\subsection{ECS Scaling}
\begin{itemize}
	\item Service CPU usage which is a CW metric, triggers CW alarm, autoscale
	\item SQS queue, polls for messages, if queue length exceeds limit (CW metric), triggers an alarm, autoscale
\end{itemize}

\subsection{ECS Rolling Updates}
\begin{itemize}
	\item we can control how many tasks can be started and stopped, and in which order
	\item EX: min 50 percent, max 100 percent, starting number of tasks: 4
\end{itemize}

\section{ECS Services and Tasks}
\begin{itemize}
	\item it is common to spin up an ECS cluster with a running service that connects to an ALB
	\item when launching a container, it will get a random port assigned to it
	\item ALB has dynamic port mapping feature - supports finding the right port on your EC2 instances
	\item You can run multiple instances of the same container onto the same EC2 instance and have them exposed thru the same ALB. 
	\item You must allow on the EC2 instance SG any port from the ALB SG
\end{itemize}

\subsection{Load Balancing for Fargate}
\begin{itemize}
	\item Fargate creates ENI for every task run. Each ENI has a unique IP but port remains the same. 
	\item Must allow on the ENI SG the task port from the ALB SG
\end{itemize}

\subsection{ECS tasks invoked by Event Bridge}
\begin{itemize}
	\item allows you to invoke an ECS task, like by an event bridge or cloud watch event
	\item EX: Fargate cluster, whenever a user uploads an object into S3 bucket, run a fargate task to process that object and insert some metadata into dynamoDB
	\item Create AWS Event Bridge event from S3 bucket. The event has a rule as a target which is to run an ECS task. Make sure the task has the proper task role. 
\end{itemize}

\section{ECR}
\begin{itemize}
	\item Elastic Container Registry
	\item Store, manage, and deploy containers on AWS, pay for what you use. 
\end{itemize}

\section{EKS}
\begin{itemize}
	\item Elastic kubernetes service
	\item launch managed kubernetes clusters on AWS
	\item kubernetes - deployment, scaling, and management of containerized applications
	\item alternate to ECS. best if you already use kubernetes on-prem and want to migrate to AWS
	\item EKS pods are like ECS tasks. They run on EKS nodes. The nodes can be managed by an ASG. 
\end{itemize}

\end{document}
