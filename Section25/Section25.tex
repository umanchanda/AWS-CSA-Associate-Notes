\documentclass[]{scrartcl}

%opening
\title{Networking - VPC}
\author{Uday Manchanda}

\begin{document}

\maketitle

\section{CIDR, Private vs Public IP}
\begin{itemize}
	\item CIDR - classless inter-domain routing
	\item Used for SG rules or networking in general
	\item Define an IP address range. EX: 0.0.0.0/0 = all ip addresses
	\item Two components: base ip (xx.xx.xx.xx) and subnet mask (/y)
	\item base ip represents an ip contained in the range
	\item subnet masks defines how many bits can change in the IP
\end{itemize}

\subsection{Subnet Masks}
\begin{itemize}
	\item allows part of the underlying IP to get additional next values from the base IP
	\item /32 allows for 1 IP = $2^0$
	\item /31 allows for 2 IP = $2^1$
	\item /30 = 4 IP = $2^2$
	\item /29 = 8, /28 = 16, /16 = 65,536
	\item /y = $2^{32-y}$
	\item /32 = no IP can change, /24 = last IP number can change, /16 = last two IP numbers can change, etc
	\item 192.168.0.0/24 = 192.168.0.0 - 192.168.0.255
	\item 192.168.0.0/16 = 192.168.0.0 - 192.168.255.255
	\item 134.56.78.123/32 = 134.56.78.123
	\item Private IP ranges: 10.0.0.0/8 (big networks), 172.16.0.0/12 (default AWS), 192.168.0.0/16 (home networks) 
\end{itemize}

\section{VPC}
\begin{itemize}
	\item VPC = virtual private cloud
	\item Can have multiple VPCs in a region (max 5)
	\item Max CIDR per VPC is 5. For each CIDR:
	\begin{itemize}
		\item Min size is /28 = 16 IPs
		\item Max is /16 = 65526 IPs
	\end{itemize}
	\item Since the VPC is private, only the private IP ranges are allowed
\end{itemize}

\subsection{Default VPC}
\begin{itemize}
	\item All new accounts have a default VPC
	\item New instances are launched into the default VPC if no subnet is specified
	\item Default VPC have internet connectivity and all instances have a public IP
	\item Also get public and private DNS name
\end{itemize}

\subsection{VPC Peering}
\begin{itemize}
	\item connect 2 VPCs privately using the AWS network
	\item makes them behave as if they were in the same network
	\item must not have overlapping CIDR
	\item is not transitive. if A and B are connected, and B and C are connected, A and C are not connected
	\item can do vpc peering with another aws account
	\item must update route tables in each VPCs subnets to ensure instances can communicate
	\item can reference a SG of a peered VPC (for cross account)
\end{itemize}

\subsection{VPC Endpoints}
\begin{itemize}
	\item meant for you to access AWS services within a private network
	\item via a NACL
	\item no need to setup all the infrastructure
	\item two types of endpoints
	\begin{itemize}
		\item interface: provisions an ENI as an entry point - most AWS services
		\item gateway: provisions a target and must be used in a route table (S3, DynamoDB)
	\end{itemize}
	\item if there are issues: check DNS setting resolution in VPC or check route tables
\end{itemize}

\subsection{VPC Flow Logs}
\begin{itemize}
	\item help you capture information about ip traffic going to your interfaces
	\item VPC flow logs, subnet flow logs, and ENI flow logs
	\item data can go to S3/CW logs
\end{itemize}

\section{Subnet}
\begin{itemize}
	\item tied to specific AZs
	\item 5 IPs are reserved in each subnet (first 4 and last one)
	\item If you need 29 IP addresses for EC2 instances you can't choose a subnet size of /27 (32 IP) since 5 IPs are reserved
	\item Need at least 64 IPs, so go for /26
\end{itemize}

\section{Internet Gateways and Route Tables}
\begin{itemize}
	\item Help our VPC instances connect with the internet
	\item created separately from VPC
	\item One VPC can only be attached to one IGW and vice versa
	\item also a NAT for the instances that have a public ipv4
	\item do not allow internet access on their own, must edit route tables
	\item if you want an ec2 to have access to the internet
	\begin{itemize}
		\item edit route table to point to the gateway for a specific IP range
		\item ec2 will then get routed directly into the gateway
	\end{itemize}
\end{itemize}

\section{NAT Instances}
\begin{itemize}
	\item allows instances in the private subnets to connect to the internet
	\item network address translation
	\item must be launched in a public subnet
	\item must have elastic IP attached
	\item route table must be configured to route traffic from private subnets to NAT instance
\end{itemize}

\section{NAT Gateways}
\begin{itemize}
	\item AWS managed NAT, higher bandwidth, better availability, no administration
	\item created in a specific AZ, cannot be used by an instance in that subnet (only from other subnets)
	\item requires IGW
	\item no need to manage SG
\end{itemize}

\subsection{NAT Gateway with High Availability}
\begin{itemize}
	\item resilient within a single AZ
	\item must create multiple NAT Gateways in multiple AZs for fault tolerance
	\item no cross AZ failover needed because if a AZ goes down it doesn't need NAT
\end{itemize}

\section{DNS Resolution Options and R53 Private Zones}
\begin{itemize}
	\item DNS resolution in VPC
	\begin{itemize}
		\item enableDnsSupport - default true
		\item enableDnsHostname - default false
	\end{itemize}
	\item if you use custom DNS domain names in a private zone in r53, you must set both these attributes to true
\end{itemize}

\section{NACL and Security Groups}
\begin{itemize}
	\item NACL = network access control list (subnet level)
	\item incoming requests are first evalulated by the NACL inbound rules and then the SG inbound rules
	\item SGs are stateful. so if an inbound rule allows traffic the outbound does as well.
	\item NACLs are stateless. so if an inbound rule allows traffic the outbound does not necessarily. 
	\item outgoing requests are first evaluated by the outbound rule on the SG and then the outbound rule on the NACL
	\item they're like a firewall which control traffic from and to subnet
	\item default nacl allows everything inbound and outbound
	\item one nacl per subnet, new subnets are assigned default nacl
	\item nacl rules have a number, higher precedence have lower numbers. 
\end{itemize}

\section{Bastion Hosts}
\begin{itemize}
	\item Use to SSH into private instances
	\item bastion is in the public subnet which is then connected to all other private subnets
	\item bastion host should only have port 22 traffic from the IP you need
\end{itemize}

\section{Site to Site VPN}
\begin{itemize}
	\item used if you have a corporate data center
	\item customer gateway on the corporate data center side
	\item vpn gateway on the vpc side
	\item site to site vpn links the two
\end{itemize}

\subsection{Virtual Private Gateway}
\begin{itemize}
	\item VPN concentrator on the AWS side of the VPN connection
	\item VGW (virtual private gateway) is created and attached to the VPC from which you want to create the site to site VPN
\end{itemize}

\subsection{Customer Gateway}
\begin{itemize}
	\item IP address:
	\begin{itemize}
		\item use static, internet-routable IP address for you customer gateway device
		\item if behind a NAT (with NAT-T), use the public IP of the NAT
	\end{itemize}
\end{itemize}

\section{Direct Connect and Direct Connect Gateway}
\begin{itemize}
	\item DX provides a dedicated private connection from a remote network to your VPC
	\item dedicated connection must be setup between your DC and AWS DX locations
	\item Set up Virtual private gateway on VPC
	\item use direct connect gateway to connect to one or more VPCs in different regions (same acct)
	\item Connection types
	\begin{itemize}
		\item Dedicated connections - physical ethernet port dedicated to customer, connection requests are made to AWS first and completed by AWS DX partners
		\item Hosted Connections - connection requests are made via AWS DX partners
	\end{itemize}
\end{itemize}

\section{Egress Only Internet Gateway}
\begin{itemize}
	\item egress = outgoing
	\item similar to NAT but for ipv6
	\item all ipv6 addresses are public
	\item gives ipv6 instances access to internet but won't be directly reachable by the internet
\end{itemize}

\section{AWS Private Link}
\begin{itemize}
	\item aka vpc endpoint services
	\item expose services in your VPC to other VPCs. most scalable and secure way
\end{itemize}

\section{Transit Gateway}
\begin{itemize}
	\item for having transitive peering between thousands of VPC and on-premises, hub and spoke (star) connection
	\item no need to peer the VPCs individually
\end{itemize}

\end{document}
