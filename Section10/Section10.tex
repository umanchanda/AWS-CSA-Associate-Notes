\documentclass[]{scrartcl}

%opening
\title{Route53}
\author{Uday Manchanda}

\begin{document}

\maketitle

\section{Route53}
\begin{itemize}
	\item Managed DNS
	\item the most common records are:
	\begin{itemize}
		\item A: hostname to IPv4
		\item AAAA: hostname to IPv6
		\item CNAME: hostname to hostname
		\item Alias: hostname to AWS resource
	\end{itemize}
	\item Diagram for A record
	\begin{itemize}
		\item Browser sends DNS request to Route53 with hostname
		\item Route53 responds with IP
		\item Browser makes HTTP request to the IP address, application server responds with HTTP response
	\end{itemize}
	\item Can use public (you own or buy) or private domains (resolved by your instances in your VPCs)
	\item Load balancing
	\item Health checks
	\item Routing policy
	\item If you purchase a domain on godaddy and want to use it with R53: Create public hosted zone and update the 3rd party registrar NS records
\end{itemize}

\subsection{TTL}
\begin{itemize}
	\item Time to live, its a DNS record
	\item cache the response of a DNS query, don't want to overload DNS
	\item As soon as the browser sends a DNS request, it will cache it, along with the response
	\item Higher TTL means less traffic on DNS but possibility of outdated records
\end{itemize}

\subsection{CNAME vs Alias}
\begin{itemize}
	\item AWS resources expose a hostname, but you want to expose a url
	\item CNAME: points a hostname to any other hostname, only for non-root domain (ex: something.domain.com)
	\item Alias: points a hostname to an aws resource (app.mydomain.com $\rightarrow$ blabla.domain.com). Free of charge, native health check
	\item If you have a root domain use alias, if it's not a root domain you can use either (alias preferred)
\end{itemize}

\subsection{Health Checks}
\begin{itemize}
	\item if an instance is unhealthy, R53 will not send traffic to that instance
	\item instance is unhealthy if fails three HCs in a row, opposite for healthy
	\item default HC interval is 30 seconds
	\item HTTP/TCP/HTTPS health checks
	\item can integrate with CW and linked to R53 DNS queries
\end{itemize}

\section{Routing Policy}
\begin{itemize}
	\item Simple
	\begin{itemize}
		\item use to redirect to a single resource
		\item no health checks
		\item if multiple values are returned, a random record is returned
	\end{itemize}
	\item Weighted
	\begin{itemize}
		\item controls the percentage of the requests that go to a specific endpoint
		\item EX: three EC2 instances. R53 sends x percent to EC2A, y percent to EC2B, and z percent to EC2C
		\item helpful to test 1 percent of traffic on new app version for example
		\item Helpful to split traffic between regions
		\item Will create a number of A records depending on how many IP addresses are listed
	\end{itemize}
	\item Latency
	\begin{itemize}
		\item Redirect to the server that has the least latency close to us
		\item Latency is evaluated in terms of user to designated AWS region
	\end{itemize}
	\item Failover
	\begin{itemize}
		\item Two EC2 instances, primary and secondary
		\item HC on the primary instance, if it fails, it will failover to the secondary
	\end{itemize}
	\item Geolocation
	\begin{itemize}
		\item Different from latency, routing based on user location
		\item here we specify: traffic from the UK should go to this specific IP
		\item Should create a default policy in case there's no match on the location
	\end{itemize}
	\item Geoproximity
	\begin{itemize}
		\item Route traffic to your resources based on geographic location of users and resources
		\item Ability to shift more traffic to resources based on the defined bias
		\item Reosurces can be AWS (like AWS region) or not (like Lat/Long)
		\item Specify bias values to change the size of the geographic location
		\item Useful if you need to shift resources to a certain region or vice versa
	\end{itemize}
	\item Multivalue
	\begin{itemize}
		\item Route traffic to multiple resources
		\item Want to associate a R53 HC with records
		\item Up to 8 healthy records are returned for each multivalue query
		\item Not a subsitute for ELB
	\end{itemize}
\end{itemize}


\end{document}
