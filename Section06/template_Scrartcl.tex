\documentclass[]{scrartcl}

%opening
\title{EC2 - Solutions Architect Associate Level}
\author{Uday Manchanda}

\begin{document}

\maketitle

\section{Private vs Public vs Elastic IP}
\begin{itemize}
	\item Public IP - accessible over the public Internet, can be identified on the internet, must be unique
	\item Private IP - only accessible within your private network, must be unique in that private network
	\item IPs in private network connect to internet via NAT and internet gateway (proxy)
	\item Elastic IP - when you stop and start an ec2 instance it changes its IP
	\item Can mask the failure of an instance or software by rapidly remapping the address to another instance in your acct
	\item Avoid elastic IP - use random public IP and register DNS name to it, or use a ELB and no public IP
\end{itemize}

\section{EC2 Placement Groups}
\begin{itemize}
	\item If you want control over the EC2 Instance placement strategy
	\item When creating a placement group, you specify one of the following strategies for the group:
	\begin{itemize}
		\item Cluster: clusters instances into a low latency group in a single AZ. Use case: Big data job that needs to complete fast. 
		\item Spread: spreads instances across underlying hardware (critical applications). Minimized risk, maximize high availability. 
		\item Partition: spreads instances across many different partitions within an AZ. Scale to 100s of EC2 instances per group (Hadoop,Cassandra, Kafka). 
	\end{itemize}
\end{itemize}

\section{Elastic Network Interface}
\begin{itemize}
	\item Logical component in a VPC that represents a virtual network card
	\item Provides an EC2 instance with network connectivity
	\item They have:
	\begin{itemize}
		\item Primary private IPv4 and one or more secondary IPv4
		\item One elastic IP
		\item One public IP
		\item one or more SGs
		\item MAC address
	\end{itemize}
	\item Can create an ENI independently and attach them on the fly or move them for failover
\end{itemize}

\section{EC2 Hibernate}
\begin{itemize}
	\item Stop instance: data on disk (EBS) is kept intact in the next start
	\item Terminate: any EBS volumes (root) also set-up to be destroyed is lost
	\item On start the following happens:
	\begin{itemize}
		\item First start: the OS boots and EC2 user data script is run
		\item Following starts: the OS boots up
		\item Application starts, caches get warmed up, and that can take time
	\end{itemize}
	\item Hibernate:
	\begin{itemize}
		\item in-memory RAM is preserved
		\item instance boot is much faster
		\item RAM state is written to a file in teh root EBS volume
		\item that EBS volume must be encrypted
	\end{itemize}
\end{itemize}

\section{Advanced Concepts}

\subsection{EC2 Nitro}
\begin{itemize}
	\item Underlying platform for next generation of EC2 instances
	\item New Virtualization technology - better performance like networking, \textbf{higher speed EBS}
	\item Better security
\end{itemize}

\subsection{vCPU}
\begin{itemize}
	\item core = CPU
	\item Multiple threads can run on one CPU
	\item Each thread is represented as a virtual CPU
	\item m5.2xlarge. 4 CPU, 2 threads per CPU = 8 vCPU
\end{itemize}

\subsection{Capacity Reservations}
\begin{itemize}
	\item Ensure you have EC2 capacity when needed
	\item Capacity access is immediate, planned end-date for reservation
\end{itemize}

\end{document}
