\documentclass[]{scrartcl}

%opening
\title{EC2 Instance Storage}
\author{Uday Manchanda}

\begin{document}

\maketitle

\section{EBS}
\begin{itemize}
	\item Elastic Block Store
	\item Network drive you can attach to your instances while they run
	\item allows your instances to persist data, even after their termination
	\item only mounted to one instance at a time, bound to a specific AZ
	\item Like a network USB stick
	\item uses the network to communicate the instance, which means there might be a bit of latency
	\item can be de-attached and re-attached quickly
	\item provisioned capacity
	\item Delete on termination attribute - controls the EBS behavior when an EC2 instance terminates
	\begin{itemize}
		\item by default the root EBS volume is deleted
		\item for other volumes, it is not enabled
	\end{itemize}
\end{itemize}

\subsection{EBS Snapshots}
\begin{itemize}
	\item make a backup (snapshot) of your EBS volume at a point in time
	\item recommended to de-attach volume first
	\item can copy snapshots across AZ or region
\end{itemize}

\subsection{EBS Volume Types}
\begin{itemize}
	\item 6 types
	\begin{itemize}
		\item gp2/gp3 (SSD) - general purpose volume, can be used as boot volume
		\item io1/io2 (SSD) - high performance (mission critical), can be used as boot volume
		\item st1 (HDD) - low cost, designed for frequently accessed, throughput-intensive workloads
		\item sc1 (HDD) - lowest cost, less frequently accessed workloads
	\end{itemize}
	\item characterized in size, throughput, IOPS (i/o ops per sec)
	\item Provisioned IOPS (PIOPS) - for business critical apps with sustained IOPS performance
\end{itemize}

\subsection{EBS Multi Attach}
\begin{itemize}
	\item Attach the same EBS volume to multiple EC2 instances in the same AZ
	\item Achieve higher application availability in clustered linux apps (ex: teradata)
\end{itemize}

\subsection{EBS Encryption}
\begin{itemize}
	\item Data at rest is encrypted inside the volume
	\item Data in flight moving between the instance and volume is encrypted
	\item All snapshots are encrypted
	\item To encrypt an unencrypted EBS volume:
	\begin{itemize}
		\item Create snapshot
		\item encrypt the snapshot
		\item Create new EBS volume from that snapshot
		\item attach it to the original instance
	\end{itemize}
\end{itemize}

\subsection{EBS RAID}
\begin{itemize}
	\item used if you want to massively increase IOPS or mirror your EBS volumes
	\item Options: RAID0 and 1, 5/6 (not recommended for EBS)
	\item RAID0 - increase performance. combine 2 or more volumes and getting the total disk space and I/O. If disk fails all the data is failed. 
	\item RAID1 - increase fault tolerance. mirror a volume to another. 
\end{itemize}

\section{AMI}
\begin{itemize}
	\item amazon machine image, customization of an EC2 instance
	\item faster boot/config time because all your software is pre-packaged
	\item built for a specific region (and can be copied)
	\item public AMI - maintained by AWS, your own AMI, or AMI marketplace
	\item build AMI - will create EBS snapshots
\end{itemize}

\section{EC2 Instance Store}
\begin{itemize}
	\item EBS volumes are network drives with good but limited performance
	\item If you need a high performance hardware disk, use EC2 instance store
	\item Its a physical drive attached to the server
	\item Lose their storage if they're stopped
	\item Good for buffer/cache/scratch data/temp content. Not for long term storage
	\item Risk of data loss if hardware fails
\end{itemize}

\section{EFS}
\begin{itemize}
	\item Elastic File system
	\item Managed NFS (network file system) that can be mounted on many EC2
	\item works with EC2 instances in multi AZ
	\item pay per use
	\item attach SG to EFS, then mount 1 or more ec2 instances. each instance has access to the same data
	\item use cases: content mgmt, web serving, data sharing, wordpress
	\item file system scales automatically
	\item Performance mode: higher latency, throughput, highly parallel
	\item Throughput mode: bursting, provisioned - set your throughput regardless of storage size
	\item Storage tiers: lifecycle mgmt feature - move file after N days, generally for frequently accessed files
\end{itemize}

\section{EFS vs EBS}
\begin{itemize}
	\item EBS
	\begin{itemize}
		\item can be attached to only 1 instance at a time
		\item locked at the AZ level
		\item root EBS volumes get terminated by default
		\item network volume
	\end{itemize}
	\item EFS
	\begin{itemize}
		\item mounted to several EC2 instances, only on linux instances
		\item more expensive, but you only get billed for what you use
		\item network file system
	\end{itemize}
\end{itemize}

\end{document}
